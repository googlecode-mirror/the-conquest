% Gameplay

\subsection{Game mechanics}

Throughout the following sections the main aspects of \game\ game mechanics will be
profusely discussed with the purpose of describing in an accurate way how
players will interact with the game.\\

% Armies & civilizations (bonuses)
\subsubsection{Armies and civilizations}

In every battle, the player will command an army from a given civilization based
on a concrete historical period no matter if it is a quick battle, a complete
campaign or an online encounter. Each player's army will be identified thanks
to its colour scheme which can be blue, red, green or yellow. The colour scheme
has nothing to do with the chosen civilization.\\

Each civilization will have access to a possibly different set of units. For
example, the S.XVIII Spanish navy might have a warship the French can't produce
and so on. On top of that, a given civilization may be specialized in some
kind of units so they have greater qualities (movement, attack or defense).\\

% Map
\subsubsection{Maps}

    % Victory objectives

Maps or battles (synonyms) are \game's base. When a player starts a map, two
armies will fight to obtain the victory. This goal can be achieved either by
killing all enemy units or by conquering the opponent's Operations center. Maps
will be tiled based, that is, they will be composed of a big mesh of squares.\\
 
    % Unit movement

Players will command their respective army during time limited turns. When it's
a player's turn, he can either move his units, command them to attack, conquer
enemy facilities or ask for reinforcements. When a unit is selected, a map
region will be highlighted showing the unit's reach for that turn. The distance
a given unit can walk in a turn will be determined by its properties and the
terrain it has to traverse. A unit that walks all its moving distance can still
attack the enemy in the same turn.\\

    % Unit attack

Player's units can attack the enemy if the second ones are within the attack
range of the first ones. It's important to mention that there are long
distance units (eg. artillery) and close combat units (eg. infantry). Although
melee units have to be in an adjacent position in the map in order to attack an
enemy, distance units reach the target from a longer distance. To attack, the
player has to select the attacking unit and every reachable enemy unit will be
highlighted, then the user has to select the objective and confirm the action.
When a unit attacks the enemy it becomes unavailable for the rest of the turn.\\

    % Territory control

Players will struggle to control more and more territory as the game evolves.
There will be several cities and facilities spread over the map and units can
take them. At the start of each turn, armies will receive certain quantity of
resources depending on how many buildings are under their control. To take a
city or facility, the player has to move a unit to the correspondent map
position and select the \emph{conquer} option.\\

    % Reinforcements

A commander can ask for reinforcements if his army has enough resources. To
gather new units the player must select a special building called barrack
(only if it's already conquered) and choose the desired unit type. Each unit
has a cost proportional to its power. A barrack can produce only one unit per
turn.\\

    % Vision

When a battle starts the map is covered with war fog which means that it's
possible to see the terrain type in each area but enemy units are hidden. The
player will only be able to see enemy units if they're within the vision range
of their own units. Vision range is a unit property, so different units may
have different vision ranges.\\


% User login & scoreboards
\subsubsection{User login and scoreboards}

\game\ will have a strong competitive component thanks to global scoreboards.
If the player wants to access this feature he will have to register with a
nickname, e-mail address and password from the game menu itself. When an
official campaign map ends, the user will receive a score, if he's logged on, the
game will submit it to a server. With this information is possible to check the
best 10 players in the whole world of \game.\\

If two registered players play in the online multiplayer mode the result will
be submitted to the server as well. Every player would be able to see their
performance in terms of victories, defeats and ratio victories/defeats among
the best players in the world scores.\\

% Campaign & difficulty
\subsubsection{Campaigns and difficulty}

% Online multiplayer rules
\subsubsection{Online multiplayer rules}


\subsection{Comprised historical period}

\subsection{Units}

\subsection{Terrain and buildings}
