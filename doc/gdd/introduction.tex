% Introduction

\game\ is a 2D open source turn based historical strategy game for
Windows, GNU/Linux and Mac in which two armies have to fight on a
battlefield. The users will have to command their units in an effective
way taking the enviroment into account to defeat the oponent.\\

% Basic concepts
\subsection{Basic concepts}

The game will be build upon these fundamental ideas:

\begin{itemize}
    \item \textbf{Turn based} each player will have a limited amount of time to
    move his troops and attack enemy units.
    \item \textbf{Tiled battlefield} the combat arena will be composed of squared
    tiled of different kinds of terrain: grass, mountains, rivers, roads,
    etc. Each type of terrain will affect unit movement in different ways.
    \item \textbf{Units} different types of units including: infantry, cavalry,
    artillery and navy. Each one will have its own properties.
    \item \textbf{War fog} the area not covered by the player's unit will be covered
    in war fog. Enemy units 
    \item \textbf{Independent engine} users will be able to add maps, campaigns
    or civilizations based on the historical age they want.
    \item \textbf{Quick battle} combat against the IA in the chosen map.
    \item \textbf{Campaign mode} several battles in linear order against the IA
    inspired by real events.
    \item \textbf{Online multiplayer} combat against a human opponent through the
    network.
    \item \textbf{Scoreboards} global rankings that shows players achievements.
    \item \textbf{Historical info} possibility of visualize historical information
    related to the active map, unit or civilization.
\end{itemize}

% Target
\subsection{Target users and platforms}

\game\ is aimed at a wide range of users varying from young gamers
in school age (12 years old) to more experienced users (30 years old).
The game allows the player to learn interesting information about real
battles while offers attractive mechanics. \game\ can be quite useful for
students when they're learning a historical age that appears in the game.
Experienced users will find in its online competitive features the appeal
to play.\\

Given is a 2D game the technical requirements aren't excessive. In order
to play \game\ properly, the user should have:

\begin{itemize}
    \item \textbf{Operating System}: Windows, GNU/Linux, Mac.
    \item \textbf{Processor}: 1GHz single core.
    \item \textbf{RAM}: 100MB.
    \item \textbf{Hard disk}: 100MB.
    \item \textbf{Video card}: hardware accelerated with 100MB.
    \item \textbf{Others}: keyboard, mouse and Internet connection.
\end{itemize}

\textbf{Important note}: these technical requirements are prone to vary as the
project matures.

% Visual style
\subsection{Visual style}

\game\ will have a cartoon look using a bright colour palette and thick
black borders. This decision comes from the necessity of reducing the screen
violence, this way the project could look for support in educational Linux distributions
such as Guadalinex. A friendly look will attract a wider age range as well.\\

A high level of detail is not required, in fact, simplistic characters are
preferred. However, each unit type (cavalry, infantry, artillery or navy) must be easily
recognisable. For each set of units there must be available several uniform
colour schemes so different armies can be distinguishable, at first glance.

